% Options for packages loaded elsewhere
% Options for packages loaded elsewhere
\PassOptionsToPackage{unicode}{hyperref}
\PassOptionsToPackage{hyphens}{url}
\PassOptionsToPackage{dvipsnames,svgnames,x11names}{xcolor}
%
\documentclass[
  letterpaper,
  DIV=11,
  numbers=noendperiod]{scrartcl}
\usepackage{xcolor}
\usepackage{amsmath,amssymb}
\setcounter{secnumdepth}{-\maxdimen} % remove section numbering
\usepackage{iftex}
\ifPDFTeX
  \usepackage[T1]{fontenc}
  \usepackage[utf8]{inputenc}
  \usepackage{textcomp} % provide euro and other symbols
\else % if luatex or xetex
  \usepackage{unicode-math} % this also loads fontspec
  \defaultfontfeatures{Scale=MatchLowercase}
  \defaultfontfeatures[\rmfamily]{Ligatures=TeX,Scale=1}
\fi
\usepackage{lmodern}
\ifPDFTeX\else
  % xetex/luatex font selection
\fi
% Use upquote if available, for straight quotes in verbatim environments
\IfFileExists{upquote.sty}{\usepackage{upquote}}{}
\IfFileExists{microtype.sty}{% use microtype if available
  \usepackage[]{microtype}
  \UseMicrotypeSet[protrusion]{basicmath} % disable protrusion for tt fonts
}{}
\makeatletter
\@ifundefined{KOMAClassName}{% if non-KOMA class
  \IfFileExists{parskip.sty}{%
    \usepackage{parskip}
  }{% else
    \setlength{\parindent}{0pt}
    \setlength{\parskip}{6pt plus 2pt minus 1pt}}
}{% if KOMA class
  \KOMAoptions{parskip=half}}
\makeatother
% Make \paragraph and \subparagraph free-standing
\makeatletter
\ifx\paragraph\undefined\else
  \let\oldparagraph\paragraph
  \renewcommand{\paragraph}{
    \@ifstar
      \xxxParagraphStar
      \xxxParagraphNoStar
  }
  \newcommand{\xxxParagraphStar}[1]{\oldparagraph*{#1}\mbox{}}
  \newcommand{\xxxParagraphNoStar}[1]{\oldparagraph{#1}\mbox{}}
\fi
\ifx\subparagraph\undefined\else
  \let\oldsubparagraph\subparagraph
  \renewcommand{\subparagraph}{
    \@ifstar
      \xxxSubParagraphStar
      \xxxSubParagraphNoStar
  }
  \newcommand{\xxxSubParagraphStar}[1]{\oldsubparagraph*{#1}\mbox{}}
  \newcommand{\xxxSubParagraphNoStar}[1]{\oldsubparagraph{#1}\mbox{}}
\fi
\makeatother


\usepackage{longtable,booktabs,array}
\usepackage{calc} % for calculating minipage widths
% Correct order of tables after \paragraph or \subparagraph
\usepackage{etoolbox}
\makeatletter
\patchcmd\longtable{\par}{\if@noskipsec\mbox{}\fi\par}{}{}
\makeatother
% Allow footnotes in longtable head/foot
\IfFileExists{footnotehyper.sty}{\usepackage{footnotehyper}}{\usepackage{footnote}}
\makesavenoteenv{longtable}
\usepackage{graphicx}
\makeatletter
\newsavebox\pandoc@box
\newcommand*\pandocbounded[1]{% scales image to fit in text height/width
  \sbox\pandoc@box{#1}%
  \Gscale@div\@tempa{\textheight}{\dimexpr\ht\pandoc@box+\dp\pandoc@box\relax}%
  \Gscale@div\@tempb{\linewidth}{\wd\pandoc@box}%
  \ifdim\@tempb\p@<\@tempa\p@\let\@tempa\@tempb\fi% select the smaller of both
  \ifdim\@tempa\p@<\p@\scalebox{\@tempa}{\usebox\pandoc@box}%
  \else\usebox{\pandoc@box}%
  \fi%
}
% Set default figure placement to htbp
\def\fps@figure{htbp}
\makeatother





\setlength{\emergencystretch}{3em} % prevent overfull lines

\providecommand{\tightlist}{%
  \setlength{\itemsep}{0pt}\setlength{\parskip}{0pt}}



 


\usepackage{sectsty}
\usepackage{xcolor}
\definecolor{primaryorange}{HTML}{E06E53}
\definecolor{secondaryorange}{HTML}{B8573E}
\sectionfont{\color{primaryorange}\sffamily}
\subsectionfont{\color{secondaryorange}\sffamily}
\subsubsectionfont{\color{secondaryorange}\sffamily}
\let\maketitle\relax
\KOMAoption{captions}{tableheading}
\makeatletter
\@ifpackageloaded{caption}{}{\usepackage{caption}}
\AtBeginDocument{%
\ifdefined\contentsname
  \renewcommand*\contentsname{Table of contents}
\else
  \newcommand\contentsname{Table of contents}
\fi
\ifdefined\listfigurename
  \renewcommand*\listfigurename{List of Figures}
\else
  \newcommand\listfigurename{List of Figures}
\fi
\ifdefined\listtablename
  \renewcommand*\listtablename{List of Tables}
\else
  \newcommand\listtablename{List of Tables}
\fi
\ifdefined\figurename
  \renewcommand*\figurename{Figure}
\else
  \newcommand\figurename{Figure}
\fi
\ifdefined\tablename
  \renewcommand*\tablename{Table}
\else
  \newcommand\tablename{Table}
\fi
}
\@ifpackageloaded{float}{}{\usepackage{float}}
\floatstyle{ruled}
\@ifundefined{c@chapter}{\newfloat{codelisting}{h}{lop}}{\newfloat{codelisting}{h}{lop}[chapter]}
\floatname{codelisting}{Listing}
\newcommand*\listoflistings{\listof{codelisting}{List of Listings}}
\makeatother
\makeatletter
\makeatother
\makeatletter
\@ifpackageloaded{caption}{}{\usepackage{caption}}
\@ifpackageloaded{subcaption}{}{\usepackage{subcaption}}
\makeatother
\usepackage{bookmark}
\IfFileExists{xurl.sty}{\usepackage{xurl}}{} % add URL line breaks if available
\urlstyle{same}
\hypersetup{
  pdftitle={ECOFLOR 2026: Book of Abstracts},
  colorlinks=true,
  linkcolor={primaryorange},
  filecolor={primaryorange},
  citecolor={primaryorange},
  urlcolor={primaryorange},
  pdfcreator={LaTeX via pandoc}}


\title{ECOFLOR 2026: Book of Abstracts}
\author{}
\date{}
\begin{document}
\maketitle


\begin{titlepage}
\centering
\vspace*{2cm}
{\Huge \textbf{ECOFLOR 2026: Book of Abstracts} \par}
\vspace{1.5cm}
\includegraphics[width=0.5\textwidth]{images/logo_without_bakground_small.png}
\vfill
{\Large \today \par}
\end{titlepage}

\newpage
\tableofcontents
\newpage

Welcome to the ECOFLOR 2026 Book of Abstracts.

\section{Floral Evolution, Breeding Systems \& Reproductive
Success}\label{floral-evolution-breeding-systems-reproductive-success}

\subsubsection{Pollinator-mediated floral evolution in the
pollination-generalised plant Viscaria
vulgaris}\label{pollinator-mediated-floral-evolution-in-the-pollination-generalised-plant-viscaria-vulgaris}

\textbf{Presenter:} Aarushi Susheel

\textbf{Affiliation:} Lund University

\textbf{Authors:} Aarushi Susheel, Felipe Torres-Vanegas, Ciara Dwyer,
Yedra Garcia, Sophie Hecht, Magne Friberg \& Øystein H. Opedal

Pollinator-mediated selection can lead to large variation in floral
traits. This has been well researched in specialist systems, where one
pollinator species interacts with a flowering species. In generalist
systems, where one flowering plant interacts with several pollinator
species, changes in the size and composition of the pollinator community
can alter the patterns of selection acting on the plant. Through my PhD,
I will study how a pollination-generalised flowering plant, Viscaria
vulgaris, adapts to a functionally diverse pollinator community that
varies both spatially and temporally. The study involves measurement of
plant and pollinator phenotypes, pollinator visitation, pollinator
effectiveness, and plant fitness. Combining these data with selection
studies across multiple years in multiple populations, I aim to quantify
the importance of functionally distinct pollinators in pollination and
floral divergence. Initial data analysis has revealed functionally
diverse pollinator assemblages within each plant population, along with
evidence for phenotypic selection on floral traits. I plan to link these
patterns by presenting findings from single-visit efficiency experiments
and pollinator visitation rates to quantify the `importance' of each
pollinator in the local pollinator community of various populations.
This would pave the way for constructing models that will assess the
impact of functionally diverse pollinator assemblages on floral
evolution.

\begin{center}\rule{0.5\linewidth}{0.5pt}\end{center}

\subsubsection{When is selfing not an evolutionary
dead-end?}\label{when-is-selfing-not-an-evolutionary-dead-end}

\textbf{Presenter:} Øystein Opedal

\textbf{Affiliation:} Lund University

\textbf{Authors:} Øystein Opedal \& Josselin Clo

Partial self-fertilization is a common reproductive strategy in
flowering plants with important consequences for population demography
and phenotypic evolution. While selfing has traditionally been thought
to reduce adaptive potential, recent insights into the genetics of
selfing and the ecology of plant mating challenge this long-standing
idea. We discuss the current state of Stebbins' dead-end hypothesis
through the lens of recent quantitative-genetic models, and by
considering variation in mating systems and evolutionary potential
across species' ranges. While macroevolutionary transitions tend to
proceed from outcrossing towards selfing, outcrossing rates appear to
evolve readily among populations and congeners. There is limited
evidence that selfing reduces adaptive potential (evolvability) as
reflected in standing genetic variation of phenotypic traits, and that
evolvability is reduced in the kinds of harsh environments often
occupied by selfers. Although reversals from near-complete selfing
toward outcrossing may remain challenging due to the population-genetic
properties of selfing lineages, we propose that populations of
mixed-mating species are often able to track variation in their
reproductive environment through evolutionary changes in selfing rate.
Thus, self-fertilization may not always represent an evolutionary dead
end. We conclude by outlining a modern quantitative-genetic research
programme aimed at better understanding the microevolutionary dynamics
of plant mating systems.

\begin{center}\rule{0.5\linewidth}{0.5pt}\end{center}

\subsubsection{Pollinators and Plant Rarity: Variation in Pollinator
Interactions and Reproductive Systems Across Rare and Widespread Plant
Species}\label{pollinators-and-plant-rarity-variation-in-pollinator-interactions-and-reproductive-systems-across-rare-and-widespread-plant-species}

\textbf{Presenter:} Sara M. Brito Lopes

\textbf{Affiliation:} Centre for Functional Ecology, University of
Coimbra (CFE - UC)

\textbf{Authors:} Sara Brito Lopes, Hugo Gaspar, Pedro Lopes, Afonso
Petronilho, Ana Afonso, Catarina Siopa, João Loureiro, Sílvia Castro

The current human-driven loss of biodiversity is accelerating, and
plants are among the most affected groups, with an estimated 45\% of
angiosperms at risk of extinction. Understanding the drivers of rarity
and threat is essential for predicting species persistence and for
designing effective conservation plans that prioritise vulnerable taxa.
Most angiosperms rely on animal pollination for successful reproduction
making pollinator interactions a key component of their success.
However, information on pollinator identity remains scarce for many
species. In this study, we selected 14 congeneric species pairs, each
consisting of one rare species (with varying conservation statuses) and
one widespread relative. For all species, we conducted pollinator
censuses and net-sweeping surveys to characterise their pollinator
communities. We also performed controlled pollination experiments,
including pollen supplementation and pollinator exclusion, to evaluate
pollen limitation and pollinator dependence, respectively. Additionally,
we quantified pollen/ovule ratios to infer the breeding system. With
these data, it was possible to identify the pollinators of the studied
species and to evaluate differences in visitation rates, pollinator
diversity, pollen limitation and pollinator dependence between rare and
widespread taxa, as well as among conservation categories. These
insights are particularly relevant to inform the development of
effective, evidence-based conservation strategies.

\begin{center}\rule{0.5\linewidth}{0.5pt}\end{center}

\subsubsection{Integrated fitness pathways expose concealed habitat-loss
impacts in sexually deceptive
orchid}\label{integrated-fitness-pathways-expose-concealed-habitat-loss-impacts-in-sexually-deceptive-orchid}

\textbf{Presenter:} Joshua Borràs

\textbf{Affiliation:} University of the Balearic Islands (UIB)

\textbf{Authors:} Joshua Borràs, Miquel Capó Yedra García, Øystein H.
Opedal, Amparo Lázaro and Joana Cursach

Natural habitat loss is one of the main threats to biodiversity.
Understanding how landscape degradation affects reproductive success is
essential for plant conservation, especially for species involved in
specialized pollination systems. We evaluated how habitat loss
influences reproductive fitness in the sexually deceptive Ophrys
balearica using two years of data from six populations, three in
conserved and three in disturbed landscapes. We quantified herbivory
affecting flowers and inflorescences, pollinaria removal and deposition,
fruit production, plant traits, and the composition of co- flowering
species. A path-analytical modelling framework tracked each reproductive
stage to assess how habitat loss, herbivory and pollination shape
reproductive output. Herbivory was the strongest constraint on
reproductive fitness. Both herbivory and pollinator visitation were
higher in disturbed landscapes, with visitation varying across sites and
years with no differences in observed fitness. Floral display increased
visitation and improved both male and female fitness components,
although morphological traits explained little fitness variation once
herbivory and visitation were accounted for. Overall, orchid populations
in disturbed landscapes showed cumulative reductions in relative fitness
when all reproductive stages were integrated. This study shows that
habitat loss alters herbivore pressure and pollinator visitation,
leading to reduced reproductive success in this sexually deceptive
orchid.

\begin{center}\rule{0.5\linewidth}{0.5pt}\end{center}

\subsubsection{Phylogeography of an invasion to track rapid floral
evolution}\label{phylogeography-of-an-invasion-to-track-rapid-floral-evolution}

\textbf{Presenter:} Maria Clara Castellanos

\textbf{Affiliation:} University of Sussex

\textbf{Authors:} Romero-Bravo, A., J. 0'Flaherty, J. Green, L. Unwin \&
M.C. Castellanos

Recent plant range expansions where pollinators change provide a unique
opportunity to study the potential and speed of floral adaptive change.
We have been studying the common foxglove, Digitalis purpurea, to
investigate convergent floral changes after the addition of hummingbirds
as pollinators when naturalised in tropical mountains. In addition to
our previous reports of morphological changes, we now have new evidence
of changes in nectar traits consistent with bird pollination. To confirm
the convergent nature of these changes, here we use a phylogeographic
approach to reconstruct the invasion of focal Colombian and Costa Rican
populations from the native European range. We used
genotyping-by-sequencing on individuals from eleven native populations
in Europe and three populations in the introduced range. Our
phylogeographic reconstruction points at Central Europe as the source of
two recent and independent introduction events to South and Central
America. Within the native range, population structure is consistent
with a historic northward expansion from southern European populations
and the colonisation of Norway from Britain across the North Sea. Our
phylogeographic analysis provides the most comprehensive insight onto
the colonisation history and the genetic relationships across
populations of Digitalis purpurea, an emerging model species to study
adaptive changes in novel pollinator environments.

\begin{center}\rule{0.5\linewidth}{0.5pt}\end{center}

\subsubsection{Beyond Seed Set: What Shapes Seed Quantity and Viability
in Wild Plant
Assemblages?}\label{beyond-seed-set-what-shapes-seed-quantity-and-viability-in-wild-plant-assemblages}

\textbf{Presenter:} Estefanía Tobajas

\textbf{Affiliation:} BC3-Basque Centre For Climate Change-Klima
Aldaketa Ikergai

\textbf{Authors:} Estefanía Tobajas, Luis J Chueca, Christian Gostout,
Brais Hermosilla, Jennifer Rose, Xabier Salgado-Irazabal, Celia
Baigorri, Montserrat Muriana, Jon Poza, Ainhoa Magrach.

Understanding the drivers of plant reproductive success is crucial for
predicting population dynamics and ecosystem functioning. Reproductive
success depends not only on the number of fruits and seeds produced, but
also on the viability of these seeds, an aspect that is seldom
considered in pollination studies. In this study, we investigated the
factors influencing both seed production and seed viability within a
diverse plant community. During 2024, we marked and collected fruits
from multiple plant species across 16 sites in Gorbea Natural Park (N
Spain), quantified their seed production, and assessed seed viability
using tetrazolium staining. Preliminary results show that plant species
richness has a positive effect on seed set, although the magnitude of
this effect varies among species. We also find evidence of a trade- off
between seed quantity and seed quality: fruits with more seeds tend to
produce a lower proportion of viable seeds. Next steps will incorporate
additional mechanisms, including temporal and resource use overlap among
plant species, functional diversity, and pollinator community structure,
to better understand the pathways shaping reproductive outcomes.
Overall, this study highlights the value of integrating seed production
and viability to understand plant reproductive success, and underscores
the influence of community composition and biotic interactions on
reproductive performance in natural plant communities.

\begin{center}\rule{0.5\linewidth}{0.5pt}\end{center}

\subsubsection{Habitat factors and traits shape plant- pollinator
interactions in a semi-arid
landscape}\label{habitat-factors-and-traits-shape-plant--pollinator-interactions-in-a-semi-arid-landscape}

\textbf{Presenter:} Diana Michael

\textbf{Affiliation:} Ashoka University, Sonipat

\textbf{Authors:} Diana Michael, Kunjan Joshi, Shivani Krishna

Plant--pollinator interactions are central to understanding ecological
processes that shape plant community reproductive success. Although
species-level interactions help predict community stability, examining
individual-level interactions of keystone species is crucial. This study
investigates how habitat factors and floral traits influence pollinator
interactions in Maytenus senegalensis, a dominant species in the
semi-arid Aravalli Hills, India. We quantified flower production, nectar
concentration, flower diameter, soil moisture, distance to habitat edge,
and proportion of co-flowering conspecifics to assess their effects on
pollinator visitation and reproductive success. We found variation in
reproductive investment and a trade-off between flower production and
reward quality: individuals producing more flowers had lower nectar
sugar concentration. High flower production negatively affected
reproductive success, likely due to increased within-plant visitation.
Eristalinus and Apis were dominant pollinators, with Dipterans playing a
key role in maintaining network stability. Higher conspecific neighbors
reduced pollen deposition, indicating competition. Individual plants
also showed varying specialization in their interaction niches,
suggesting divergence driven by pollinator-mediated competition.
Disturbances to plants with high pollinator connectance strongly
affected network stability. Overall, our results show that microhabitat
and neighborhood context shape individual interaction niches, with
allocation trade-offs and conspecific competition jointly influencing
pollination and fitness in semi-arid systems experiencing environmental
change.

\begin{center}\rule{0.5\linewidth}{0.5pt}\end{center}

\subsubsection{Gene expression plasticity across regulatory pathways for
flowering time in Arabidopsis
thaliana}\label{gene-expression-plasticity-across-regulatory-pathways-for-flowering-time-in-arabidopsis-thaliana}

\textbf{Presenter:} Patricia Roca Villanueva

\textbf{Affiliation:} University of Granada

\textbf{Authors:} Patricia Roca-Villanueva, Ana García Muñoz, Xavier
Picó.

Gene expression plasticity can be defined as the ability of a single
gene to adjust its expression in response to changes in the environment.
Understanding how environmental cues affect gene expression plasticity
in field conditions is a major challenge, which may help grasp the
complexity of genotype-phenotype relationships. Nevertheless, gene
expression plasticity in natural conditions has barely received
attention due to the logistical complexity to estimate gene expression
outdoors. In this study, we investigated how environmental conditions
modulate the expression of flowering-related genes from all known
regulatory pathways in natural accessions of Arabidopsis thaliana across
multiple timescales relevant for gene expression. Using generalized
linear mixed models (GLMMs) on a previous whole-genome gene expression
dataset obtained from locally-adapted accessions in natural conditions,
we evaluated gene expression plasticity across diurnal (morning vs.
afternoon), seasonal (across developmental stages: vegetative,
inductive, and reproductive), and annual timescales (over two different
years). Our analysis focused on a set of 306 known genes in A. thaliana
related to flowering time to estimate their expression plasticity and to
quantify the differences across accessions and various regulatory
pathways. Overall, our work provides valuable insights to understand how
genes and regulatory pathways for flowering respond to natural
environmental variation to complete the vegetative-to- reproductive
transition in plants, which is a major trait under strong selection in
annuals and short-lived perennials.

\begin{center}\rule{0.5\linewidth}{0.5pt}\end{center}

\subsubsection{Kinship, cross pollination and self-incompatibility:
exploring the fitness costs of genetic
viscosity.}\label{kinship-cross-pollination-and-self-incompatibility-exploring-the-fitness-costs-of-genetic-viscosity.}

\textbf{Presenter:} Camilo Ferrón

\textbf{Affiliation:} 1 Departamento de Biología y Geología, Física y
Química Inorgánica, Universidad Rey Juan Carlos (URJC), Móstoles, 28933,
Spain. 2 Instituto de Investigación en Cambio Global (IICG-URJC),
Universidad Rey Juan Carlos, Móstoles, 28933, Spain.

\textbf{Authors:} Camilo Ferrón, Ana García-Muñoz

How spatial genetic structure and pollen-mediated interactions shape
reproductive success and offspring fitness in a self-incompatible plant.

\begin{center}\rule{0.5\linewidth}{0.5pt}\end{center}

\section{Traits, Plasticity \& Signals}\label{traits-plasticity-signals}

\subsubsection{Flower economic spectrum: A key to understanding the
floral diversity of alpine
plants}\label{flower-economic-spectrum-a-key-to-understanding-the-floral-diversity-of-alpine-plants}

\textbf{Presenter:} Lucie Holzbachová

\textbf{Affiliation:} Charles University Prague

\textbf{Authors:} Lucie Holzbachová, Petr Sklenář, Jakub Štenc

Flowers of zoogamous plant species are subject to combined selection
pressures from abiotic and biotic factors, yet their impact on floral
diversity has mostly been studied separately. The concept of the Flower
Economic Spectrum (FES) has been recently proposed to understand the
evolutionary phenotypic variation and diversity of functional traits in
flowers that evolved under multiple selection factors. The
world\textquotesingle s mountainous regions are home to a large part of
global biodiversity and mountain environments impose strong pressures on
flowering, such as extreme climatic conditions and low abundance and
diversity of pollinators. However, not all alpine regions share the same
conditions. Tropical and temperate mountains differ in many important
ecological factors that strongly influence generative plant
reproduction. We studied more than 50 herbaceous and woody species from
two mountain regions (Ecuador and the USA). The study examines the
phenotypic variability of alpine flowers through the lens of the FES,
i.e.~relationship between flower longevity and investment into flower
biomass (cost of flower production). Our preliminary results show a
positive association between flower biomass investment and flower
longevity, with differences in longevity patterns between the two alpine
regions. Together, these findings provide empirical support for the
proposed Flower Economic Spectrum.

\begin{center}\rule{0.5\linewidth}{0.5pt}\end{center}

\subsubsection{Flower orientation influences wild pollinator behaviour:
a field study on natural and artificial
flowers}\label{flower-orientation-influences-wild-pollinator-behaviour-a-field-study-on-natural-and-artificial-flowers}

\textbf{Presenter:} Chiara Buonanno

\textbf{Affiliation:} University of Parma

\textbf{Authors:} Chiara Buonanno, Giannetti Daniele, Marta Barberis,
Marta Galloni, Donato A. Grasso

The foraging activity of bees is a complex behaviour that depends, among
other factors, on some physical features of flowers. Of particular
importance are accessibility of floral rewards, floral proportions,
symmetry and orientation. Several studies have investigated the effects
of flower orientation using colonies of bees under experimental
controlled condition. In the present study we performed field
experiments employing both artificial and natural flowers (different
species of the genus Salvia) characterized by zygomorphic symmetry. By
altering the orientation of flowers, we analysed how different species
of wild bees approached and interacted with them. The results showed
that pollinators visiting artificial flowers, especially of family
Halictidae, preferred those with a horizontal landing surface.
Concerning real flowers, several species of Apidae visited significantly
more flowers with natural orientation or those turned of 90°. Our
results, including observations on insect approach and visiting methods,
showed that even minor alterations in flower orientation can markedly
affect pollinator behaviour, providing new perspectives into the
ecological and evolutionary mechanisms shaping plant-- pollinator
interactions.

\begin{center}\rule{0.5\linewidth}{0.5pt}\end{center}

\subsubsection{Floral Trait Thermal Plasticity in a Common
Crop}\label{floral-trait-thermal-plasticity-in-a-common-crop}

\textbf{Presenter:} Lucy Unwin

\textbf{Affiliation:} University of Sussex

\textbf{Authors:} Unwin, L. A., Millerchip, E. K., Dadswell, C.,
Castellanos, M. C

Plasticity in floral traits, particularly those related to pollinator
reward and attraction, can influence both the types of pollinators that
visit a flower and the nature of those interactions. As flowers commonly
exhibit suites of traits that align with pollinator preferences,
environmentally driven (=plastic) changes in floral traits can alter
plant-pollinator interactions in both crop and wild plants. Whilst
floral nectar traits have frequently been cited as `highly plastic',
many studies do not measure true plasticity - that is, variation in
trait expression across environments within the same genotype.
Consequently, the true extent of plasticity in floral nectar traits
remains poorly understood. Understanding this is central to predicting
the resilience of plant--pollinator interactions in the face of
environmental change. We used an experimental setup to measure
plasticity in response to temperature in floral nectar volume, flower
size, and nectar sugar characteristics in the common bean Phaseolus
vulgaris L., a globally important crop in the Fabaceae family. P.
vulgaris individuals were grown in controlled greenhouse conditions,
then allowed to flower at temperatures of 16, 23, and 30°C for 3-day
periods. Individual plants experienced multiple temperature treatments
to assess plasticity in floral traits. Both nectar volume and flower
size show significant plasticity in response to temperature. For both
traits, the response to temperature was quadratic, consistent with the
presence of a thermal optimum. Interestingly, plants varied in their
baseline nectar production, but the shape of the plastic response was
highly consistent across plants, suggesting plant-level physiological
control of this trait. For flower size, plastic responses were less
consistent and there was variation across flowers within plants.
Understanding the plasticity of floral traits in crop species provides
key information on the potential to breed cultivars with stable reward
production that can benefit both yields and

pollinators.

\begin{center}\rule{0.5\linewidth}{0.5pt}\end{center}

\section{Community Ecology, Networks \& Niche
Partitioning}\label{community-ecology-networks-niche-partitioning}

\subsubsection{Effects of forest structural heterogeneity on hoverfly
diversity and pollination
potential.}\label{effects-of-forest-structural-heterogeneity-on-hoverfly-diversity-and-pollination-potential.}

\textbf{Presenter:} Clàudia Massó Estaje

\textbf{Affiliation:} University of Würzburg

\textbf{Authors:} Clàudia Massó Estaje, Anne Chao, Jörg Müller, Alice
Claßen, Ingolf Steffan-Dewenter

Habitat homogenization from intensive forest management has reduced
pollinator diversity, threatening forest regeneration and plant
reproduction. Experimental evidence on how forest structural
heterogeneity influences pollinator communities at the landscape scale,
however, remains scarce. We tested whether enhancing structural
heterogeneity through deadwood enrichment and canopy gap creation
(Enhancement of Structural Beta Complexity, ESBC) promotes hoverfly
diversity, key pollinators in temperate forests, and whether this effect
is driven by local (α) diversity or species turnover (β diversity). Our
large-scale forest experiment across 11 regions in Germany compared
paired small forest landscapes (ESBC vs.~control), comprising 234
patches sampled with pan traps across three seasons. Using
incidence-based Hill numbers, we quantified taxonomic, functional, and
phylogenetic diversity (TD, FD, PD) at α, β, and γ scales. Structurally
heterogeneous landscapes supported higher γ-diversity across all
biodiversity dimensions, particularly for taxonomic richness, suggesting
that rare hoverfly species benefit most. Most diversity gains were
driven by α rather than β components. Our findings provide experimental
evidence that enhancing forest structural heterogeneity can restore
multi-dimensional pollinator diversity, reinforcing its potential to
sustain floral visitation networks and counteract biotic homogenization.

\begin{center}\rule{0.5\linewidth}{0.5pt}\end{center}

\subsubsection{How is the buzz-pollination niche partitioned among
co-flowering
plants?}\label{how-is-the-buzz-pollination-niche-partitioned-among-co-flowering-plants}

\textbf{Presenter:} Agnes Dellinger

\textbf{Affiliation:} University of Vienna

\textbf{Authors:} Benjamin Lazarus, Agnes S. Dellinger

Co-flowering plants may overlap or diverge in pollination niche, with
traits related to pollinator attraction (e.g., color, scent) and fit
(e.g., herkogamy) regarded as particularly important in mediating
pollination niche position. Buzz-pollinated flowers are particularly
interesting in this context since they have a third, invisible and
understudied trait component determining niche position: their
vibrational properties. Buzz-pollination is a functionally highly
specialized pollination mechanism where large quantities of pollen can
only be dislodged when bees apply vibrations in the range of 100-400 Hz
to the flowers. Whether co-flowering, buzz-pollinated species are
``tuned'' to different bees, or rely on common strategies of niche
partitioning such as differential attraction and fit, remains unclear.
In my talk, I will explore these questions using community-level
plant-pollinator interaction studies of the plant family Melastomataceae
as a model. Melastomataceae are among the largest plant families
worldwide (close to 6000 species), almost exclusively buzz-pollinated
(96\% of species, adaptive plateau) and multiple species are commonly
co-flowering in diverse tropical habitats. Using comparative assessments
of plant-pollinator interactions, single visit experiments and
artificial vibration experiments (mimicking bees), we find that
co-flowering Melastomataceae often overlap in their bee visitor
assemblages, but that size-matching with bees (herkogamy) plays a
critical role in niche differentiation. Our artificial vibration
experiments further indicate that different species have different
vibration optima, and that differential ``tuning'' may indeed be an
important mechanism of pollination niche differentiation.

\begin{center}\rule{0.5\linewidth}{0.5pt}\end{center}

\subsubsection{A multidimensional approach reveals pollination niche
partitioning among terrestrial
orchids}\label{a-multidimensional-approach-reveals-pollination-niche-partitioning-among-terrestrial-orchids}

\textbf{Presenter:} Aurélien Caries

\textbf{Affiliation:} Lund University

\textbf{Authors:} Caries Aurélien, Friberg Magne, Opedal Øystein, García
García Yedra

Pollinator-mediated reproductive interactions between co-flowering
species are increasingly recognized for their role in community
structure. Pollination traits, flowering phenology and spatial
distribution are key axes of the pollination niche, yet few studies have
assessed their combined effects on community assembly. We quantified
pairwise overlap in pollination niches among 16 orchids, including
rewarding and deceptive species, using floral traits related to
pollinator attraction and pollination efficiency measured at two sites
on Öland (Sweden). We collected flowering times and spatial
co-occurrence data from a citizen-science database. At the local level,
we compared the coefficient of variation per trait between pollination
strategies (deceptive vs.~rewarding) and trait values across sites. Most
species pairs overlapped in at least one axis of the pollination niche.
Typically, species with high overlap across multiple niche dimensions
represented cases where the literature suggests pollinator niche
partitioning. Three food-deceptive species overlapped strongly in niche
space, despite sharing pollinators. While character displacement for
unmeasured traits via competition may occur, we hypothesise that trait
divergence may instead promote facilitation by maintaining pollinator
deception and increasing visitation. Our findings highlight the
complementary role of different niche dimensions in enhancing species
coexistence and support emerging evidence that deceptive orchids may
facilitate each other.

\begin{center}\rule{0.5\linewidth}{0.5pt}\end{center}

\subsubsection{The impact of Impatiens glandulifera (Himalayan Balsam)
on the pollination of the native Stachys sylvatica (Hedge Woundwort) in
the
UK}\label{the-impact-of-impatiens-glandulifera-himalayan-balsam-on-the-pollination-of-the-native-stachys-sylvatica-hedge-woundwort-in-the-uk}

\textbf{Presenter:} Samira Ben-Menni Schuler

\textbf{Affiliation:} Universidad de Granada

\textbf{Authors:} Samira Ben-Menni Schuler, Laura Mary White, George
Horn, Rocío Pérez-Barrales

Invasive plants can alter pollination dynamics by attracting shared
pollinators away from native flora, potentially reducing reproductive
success. Impatiens glandulifera (Himalayan balsam) is a widespread
invader in the UK whose large, nectar-rich flowers attract bumblebees
and may disrupt native pollination. We assessed its impact on the
pollination of the native Stachys sylvatica through (1) observations in
pristine and invaded sites and (2) an experimental introduction of I.
glandulifera into an uninvaded habitat. Across natural sites, S.
sylvatica stigmas in invaded areas received \textasciitilde3.5 times
less conspecific pollen than in pristine sites. In the introduction
experiment, the arrival of I. glandulifera caused a rapid decline in
conspecific pollen deposition, decreasing by \textasciitilde80\% within
four days, while invasive pollen appeared on up to 70\% of stigmas.
Combined visitation and pollen data indicate that behavioural diversion
of bumblebees better explains the reduction in conspecific pollen than
heterospecific pollen deposition. Our results provide experimental
evidence that I. glandulifera can swiftly disrupt native pollination
processes during early invasion stages, highlighting the vulnerability
of co-flowering natives and the need for management strategies that
limit Himalayan balsam establishment in sensitive riparian habitats.
away from native flora, potentially reducing reproductive success.
Impatiens glandulifera (Himalayan balsam) is a widespread invader in the
UK whose large, nectar-rich flowers attract bumblebees and may disrupt
native pollination. We assessed its impact on the pollination of the
native Stachys sylvatica through (1) observations in pristine and
invaded sites and (2) an experimental introduction of I. glandulifera
into an uninvaded habitat. Across natural sites, S. sylvatica stigmas in
invaded areas received \textasciitilde3.5 times less conspecific pollen
than in pristine sites. In the introduction experiment, the arrival of
I. glandulifera caused a rapid decline in conspecific pollen deposition,
decreasing by \textasciitilde80\% within four days, while invasive
pollen appeared on up to 70\% of stigmas.Combined visitation and pollen
data indicate that behavioural diversion of bumblebees better explains
the reduction in conspecific pollen than heterospecific pollen
deposition. Our results provide experimental evidence that I.
glandulifera can swiftly disrupt native pollination processes during
early invasion stages, highlighting the vulnerability of co-flowering
natives and the need for management strategies that limit Himalayan
balsam establishment in sensitive riparian habitats.

\begin{center}\rule{0.5\linewidth}{0.5pt}\end{center}

\subsubsection{Lower Disturbance Correlates with Higher Robustness and
Reduced Connectance in Plant--Pollinator Networks in Es Trenc Natural
Park
(Mallorca)}\label{lower-disturbance-correlates-with-higher-robustness-and-reduced-connectance-in-plantpollinator-networks-in-es-trenc-natural-park-mallorca}

\textbf{Presenter:} Fortunato Fulvio Bitonto

\textbf{Affiliation:} Alma Mater Studiorum - University of Bologna

\textbf{Authors:} Bitonto F. F., Serra P. E., Fuster Bejarano F. ,
Gutierrez R. , Galloni M. , Traveset A.

The Biodiversity Strategy for 2030 and the Nature Restoration Law
require EU Member States to strengthen biodiversity monitoring and
restore degraded ecosystems. In line with these, we conducted a
plant--pollinator network assessment from March to June 2023 in two
areas within the Es Trenc--Salobrar de Campos Natural Park (Mallorca,
Spain): a human-impacted and a relatively undisturbed site. Pollinators
were surveyed every two weeks along a mobile transect, collected with
hand-nets, and identified to species level. Floral resources were
evaluated using twelve randomly placed 1-m² plots per area per
monitoring day. We recorded more than 1,500 insect individuals belonging
to 120 species, including 20 bee species classified as Data Deficient,
Nearly Threatened, or Endangered in the European IUCN Red List. Floral
surveys documented over 1,300 flowering units from more than 50 plant
species, including the endangered Helianthemum caput-felis.
Network-level metrics indicated that the less disturbed site exhibited
higher ecological robustness and lower connectance compared with the
anthropized area, suggesting a more stable and resilient
plant--pollinator system. These findings will be shared with the park
authorities to help inform evidence-based conservation actions aimed at
supporting plant and pollinator communities, contributing to reducing
the information gap in the Mediterranean Basin.

\begin{center}\rule{0.5\linewidth}{0.5pt}\end{center}

\subsubsection{Modeling the structure of emerging plant-pollinator
networks in a changing
world}\label{modeling-the-structure-of-emerging-plant-pollinator-networks-in-a-changing-world}

\textbf{Presenter:} Ignasi Bartomeus

\textbf{Affiliation:} EBD-CSIC

\textbf{Authors:} Ignasi Bartomeus, Nerea Montes.

While impacts of pressure-driven species losses on interaction networks
have been identified, predicting effects of full species turnover,
particularly considering colonisation of novel species, remains a
critical

challenge. Here, we identify future turnover in species pools (lost and
novel species) of local plant- pollinator networks at the EU level and
projected future plant -- pollinator assemblages. To assess the

consequences of turnover, we use existing interaction matching models to
predict (novel) pairwise interaction probability and estimated abundance
and upscale these relationships to the community level. We use a
two-step approach that captures (i) the potential network of species
interactions and (ii), how this structure redefines the final community
dynamics, accounting for interaction rewiring due to competition.

\begin{center}\rule{0.5\linewidth}{0.5pt}\end{center}

\subsubsection{How reliable are pollinator population trends? An
interplay between duration, variability, and
autocorrelation}\label{how-reliable-are-pollinator-population-trends-an-interplay-between-duration-variability-and-autocorrelation}

\textbf{Presenter:} Julia G. de Aledo

\textbf{Affiliation:} Estación Biológica de Doñana

\textbf{Authors:} Julia G. de Aledo, François Duchenne, Ignasi Bartomeus

Pollinators are susceptible to anthropogenic influences including
climate change, habitat loss, and agricultural intensification. While
pollinators are key in providing ecosystem services, supporting
\textasciitilde85\% of wild flowering plant species, detecting
population changes remains a challenge. Existing models leave a gap in
understanding fast-lived insect dynamics. In the currently available
data, there is an over-representation of recent and short time series.
Our goal is to evaluate the degree of robustness the available data can
offer to assess trends. To do so, we analyze how statistical power and
the probability of false positives are affected by key factors:
duration, slope, stochasticity, and temporal autocorrelation.
Additionally, we aim to explore how these trends are compatible with the
expectations of stable populations by developing a null-model approach.
We found a 20\% probability of detecting false positives with the
available pollinator data. We propose practical thresholds (more than 10
years) for an acceptable statistical power (75\%) to ensure trend
inferences are robust enough. However, rigorous evaluation of trends
leads to a mismatch between the need of long-term monitoring programs
and the emergency of taking conservation actions. To shorten this
distance, we provide a framework to test the compatibility of short
observed changes with expected ecological stability of a reference
population. This framework will introduce a complementary index to help
understand the observed trends.

\begin{center}\rule{0.5\linewidth}{0.5pt}\end{center}

\subsubsection{Beetles (Coleoptera) are more than just inefficient
mess-and-soil
pollinators}\label{beetles-coleoptera-are-more-than-just-inefficient-mess-and-soil-pollinators}

\textbf{Presenter:} David Peris

\textbf{Affiliation:} Institut Botànic de Barcelona, CSIC-CMCNB

\textbf{Authors:} David Peris

Beetles (Coleoptera) are often characterized as inefficient or
incidental ``mess- and-soil'' pollinators: flower visitors that
pollinate flowers while damaging them. However, growing evidence reveals
their substantial and ancient role in the evolution of plant pollination
systems. An estimated 20\% of about 400,000 species of beetles
(Coleoptera) are flower visitors. Cantharophilous plants exhibit
traits---such as robust floral structures, thermogenesis, and strong,
often spicy or fruity scents, protogynous flowers---specifically suited
to beetle visitation. Beyond their contributions to basal angiosperms,
beetles also participate in the pollination of economically significant
crops. But more importantly, as one of the earliest insect lineages to
interact with flowering plants, beetles have driven key floral
adaptations through their diverse feeding behaviors, sensory ecology,
and morphological variation. These findings highlight the complexity of
beetle--flower mutualisms and underscore the importance of reevaluating
beetles not as inefficient pollinators, but as key evolutionary agents
that have shaped modern pollination ecology.

\begin{center}\rule{0.5\linewidth}{0.5pt}\end{center}

\subsubsection{Real world open pollinator communities shapes
plant--pollinator networks across land-use
gradients}\label{real-world-open-pollinator-communities-shapes-plantpollinator-networks-across-land-use-gradients}

\textbf{Presenter:} Nerea Montes Pérez

\textbf{Affiliation:} Estación Biológica de Doñana - CSIC

\textbf{Authors:} Nerea Montes-Perez, Francisco Rodriguez-Sanchez,
Ignasi Bartomeus

Over recent decades, agricultural intensification and habitat
fragmentation have become major drivers of declines and local
extinctions. Understanding how these pressures affect ecological
dynamics is particularly crucial for plant--pollinator interaction
networks, which sustain the essential ecosystem service of pollination.
Previous research has shown that land- use intensification reduces plant
and pollinator abundance and diversity, often favouring generalist
species. Yet most of these studies typically treat communities as closed
systems where species cannot be replaced after disturbance. This
assumption may underestimate the capacity of ecological communities to
persist and buffer environmental change. Here, we investigate how plant
and pollinator abundance, species richness and key network properties
shift along an agricultural gradient and compare these responses to
expectations under a hypothetical closed-community scenario. We sampled
plant--pollinator networks over one season across 30 sites spanning a
land-use gradient in the Doñana Protected Area. Our study provides a
framework to disentangle how community turnover and network
reconfiguration contribute to the resilience of pollination systems in
human-modified landscapes.

\begin{center}\rule{0.5\linewidth}{0.5pt}\end{center}

\section{Alpine \& Montane Ecology}\label{alpine-montane-ecology}

\subsubsection{Reproductive strategies in plants of temperate and
tropical alpine
orobiomes}\label{reproductive-strategies-in-plants-of-temperate-and-tropical-alpine-orobiomes}

\textbf{Presenter:} Alptekin Koc

\textbf{Affiliation:} Charles University

\textbf{Authors:} Alptekin Koc

Pollinator composition varies considerably between tropical and
temperate alpine areas. Due to the continuous vegetation period in the
tropical alpine and them on average existing at higher altitudes than
temperate alpine regions, their invertebrate pollinator density is far
lower and mostly consists of flies. For temperate alpine habitats during
the summer, their pollinator density is higher with a more varied pool
of available pollinators than for the tropical counterpart. The question
is: How do plants cope with these conditions? Plants do not always
depend on pollinators for their sexual reproduction. They can also be
completely autonomous by being selfers or even apomicts. With the
differences in pollinators in mind, it would be assumable that there
could be more autonomous species present in the tropical alpine compared
to the temperate alpine. I investigated this aspect by doing pollination
experiments of different treatments in the field in the tropical Andes
and the temperate alpine Rocky Mountains. The resulting seed sets I used
to determine if pollinators are essential for the local plant species
reproduction and if they are pollen limited. The results can help with
establishing focused conservational efforts for certain pollinator
groups in these unique habitats.

\begin{center}\rule{0.5\linewidth}{0.5pt}\end{center}

\subsubsection{Altitudinal variation in floral allometry and its
relationship with pollinators along an altitudinal gradient of the
tropical Andes of
Bolivia}\label{altitudinal-variation-in-floral-allometry-and-its-relationship-with-pollinators-along-an-altitudinal-gradient-of-the-tropical-andes-of-bolivia}

\textbf{Presenter:} Andrés Romero-Bravo

\textbf{Affiliation:} Lund University

\textbf{Authors:} Andrés Romero-Bravo, Øystein H. Opedal and Sissi
Lozada-Gobilard

Flower traits are shaped by breeding systems and the biotic and abiotic
factors defining the pollinator environment and may thus vary along
environmental gradients. Environmental variation along altitudinal
gradients is often associated with changes in plant and animal
diversity, making such gradients ideal systems to study variation in
flower traits and their relationship with pollinators. We measured
flower traits related to advertisement and pollinator fit in 60 plant
species and recorded their legitimate pollinators along an altitudinal
gradient (400--4400 m) in the tropical Andes of Bolivia. Flower traits
included flower size (advertisement), entrance diameter, flower length
and anther-stigma distance (fit). We tested the intra-floral modularity
hypothesis which predicts that traits regulating fit tend to be more
canalized than those involved in advertising. Specifically, we asked
whether canalization varies along the studied altitudinal gradient and
across different groups of pollinators. To do so, we compared the
allometric relationships of advertisement vs fit traits. Preliminary
results show that fit traits are indeed more canalized without any
significant change along the environmental gradient or pollinator
groups, although bee-pollinated flowers seem to be more canalized than
those relying on other pollinator groups, especially birds.

\begin{center}\rule{0.5\linewidth}{0.5pt}\end{center}

\subsubsection{The importance of aiming high: bee diversity in the
canopy
of}\label{the-importance-of-aiming-high-bee-diversity-in-the-canopy-of}

a tropical montane forest

\textbf{Presenter:} Claudia Vigano

\textbf{Affiliation:} Freiburg

\textbf{Authors:} Clàudia Massó Estaje, Anne Chao, Jörg Müller, Alice
Claßen, Ingolf Steffan-Dewenter

In the tropics, canopy tree species rely disproportionately on large
Hymenoptera for pollination, but few studies have been designed to
survey bees. This lack of data is likely due to the difficulties of
sampling dense forest layers and reaching the canopy. In our work, we
used tree-climbing techniques to access the canopy of a tropical montane
forest in southern Ecuador. There, at about 25 meters above the ground,
on the branches of ten Handroanthus chrysanthus individuals, we placed
non-scented blue vane traps to capture the diversity of bees attracted
to the flowers of this mass-flowering species. Active on alternate weeks
for 18 months, the traps collected the first documented records of bee
diversity for the area. Of more than 2,000 individuals sampled, captures
were dominated by Euglossini males. In addition, traps placed below the
canopy layer, at around 10 meters, performed significantly worse,
capturing ten times fewer individuals than the canopy traps. Our results
show that even in mega- diverse forests, canopy bee assemblages can be
dominated by a surprisingly narrow group of taxa. This underscores the
ecological importance of accounting for vertical stratification and
species' niche preferences when surveying tropical bee communities,
which is essential for guiding conservation strategies.

\begin{center}\rule{0.5\linewidth}{0.5pt}\end{center}

\subsubsection{High altitude
pollinators}\label{high-altitude-pollinators}

\textbf{Presenter:} Helena Pijálková

\textbf{Affiliation:} Faculty of Science, Charles University

\textbf{Authors:} Helena Pijálková, Tadeáš Ryšan, Lucie Holzbachová,
Jakub Štenc, Alptekin Koc, Shannon Serpa, Nyika Campbell, Petr Sklenář,
Álvaro Barragán, Sisimak Duchicela, Jiří Hadrava

Our present study compares two areas belonging to the Cordillera
mountain range. Our aim is to provide an insight into the composition of
pollinators, and which predictors might affect the seasonal variability
in the pollinator communities. The alpine environment hosts many kinds
of flowers, many of which rely on insect pollinators. Yet pollinators of
alpine environments remain historically understudied, especially in the
tropics. In the tropical Andes, highest altitudes are cold and windy,
with temperatures at night falling below 0 °C. However, these conditions
remain relatively stable throughout the year, with most prominent
changes being driven by the seasonal differences in precipitation (rainy
vs.~dry season). In contrast, temperate alpine environment of Colorado
Rocky Mountains has very short vegetational season, of about three
months, when the biota must reproduce rather quickly. During this time
of the year, the temperatures often exceed 15 °C. Due to these
differences, we can expect both alpine environments to have very
different pollinator communities. Both areas were dominated by the
Diptera, however the changes in insect composition throughout the
seasons differs greatly between the two areas, as a consequence of
seasonal fluctuations in climate conditions and availability of floral
sources.

\begin{center}\rule{0.5\linewidth}{0.5pt}\end{center}

\subsubsection{Bee sampling methods along a tropical elevational
gradient}\label{bee-sampling-methods-along-a-tropical-elevational-gradient}

\textbf{Presenter:} Pedro Alonso-Alonso

\textbf{Affiliation:} JMU Würzburg University (Germany)

\textbf{Authors:} Alonso-Alonso, Pedro

Despite their relevance as pollinators, bees (Hymenoptera: Anthophila)
are not frequently sampled in the wet tropics, where bee research is
mostly taxonomical, leaving ecology often aside. In the Neotropics, two
groups have got most of the attention, Euglossini and Meliponini, due to
their abundance, but also because ecologists know how to catch them.
Most methods for collecting bees have been tested in temperate
ecosystems and applied in tropical forests without testing their
efficiency. We studied bees along an elevational gradient in SE Peru, in
the tropical Andes aiming to understand the environmental drivers behind
their diversity and abundance patterns. During 11 months of fieldwork,
we completed 3 rounds of sampling in 26 locations. We covered the whole
gradient from the open Polylepis woodland at 3500 to the lush amazonian
Terra firme forests at 230 m asl. To optimize the bee sampling we used
multiple methods, catching bees actively during transect-walks and
attracting them with scents and also passively using different kinds of
traps. Here we discuss the success of the different methods used to
collect bees in the different kinds of forests found in the elevational
gradient of the eastern slope of the tropical Andes.

\begin{center}\rule{0.5\linewidth}{0.5pt}\end{center}

\subsubsection{Comparison of Mountain Pollination in Ecuador and the
Colorado Rocky
Mountains}\label{comparison-of-mountain-pollination-in-ecuador-and-the-colorado-rocky-mountains}

\textbf{Presenter:} Tadeáš Ryšan

\textbf{Authors:} Tadeáš Ryšan, Helena Pijálková Lucie Holzbachová,
Jakub Štenc, Shannon A. Serpa, Alptekin Koc, Petr Sklenář, Álvaro
Barragán Nyika Campbell, Sisimak Duchicela, Jiří Hadrava

Mountain ecosystems impose environmental constraints, including
temperature fluctuations, steep topography, and variable resource
availability. Despite these challenges, they support communities with
remarkably high biodiversity and a high degree of endemism. Adaptation
to environmental adversity is a key driver of this diversity: species
that persist in mountains must develop a range of physiological and
ecological traits, from generalism to narrow specialism. But how do
these environmental constraints influence the pollination relationships
between plants and their pollinators? Do pollinator networks in tropical
paramo, which have relatively stable climates but complex topography,
tend to be more diverse and specialized than those in temperate
mountains with shorter growing seasons, or is the opposite true? In our
research, we investigated flowering plant and pollinator communities
throughout the flowering season using pollination transects at Pichincha
Volcano in Ecuador and Niwot Ridge, Colorado, USA. Our goal was to
determine how pollinator networks are structured and how they are
influenced by the unique environmental challenges of high-mountain
ecosystems. Using the pollination snapshot method, we recorded several
thousand interactions over the entire growing season. exposing to
different environments, so it can be triggered by both biotic and
abiotic factors. A typical plastic response in plants occurs in response
to herbivore attack with the induction of defenses, but the role of the
herbivores as modulators of the plastic response of the plant to abiotic
conditions has been seldom studied. In this study, we experimentally
explore the effect of damage by florivores and folivores on the
occurrence and intensity of floral phenotypic plasticity of Moricandia
arvensis (Brassicaceae) under two contrasting abiotic conditions. In
nature, this mustard species blooms in two contrasting environments,
facing mild and wet conditions during spring, and hot and dry during
summer. In response to these environmental changes, the same individual
is plastic for floral traits. Our preliminary results show that plants
attacked by each type of herbivores retain the capacity to flower during
summer conditions, expressing plasticity for floral traits. These
herbivores limit the plastic response of the plant to the abiotic
conditions. This study highlights the complex interaction between biotic
and abiotic stressors and their combined effect for the evolution of
plasticity in M. arvensis.

\begin{center}\rule{0.5\linewidth}{0.5pt}\end{center}

\section{Global Change: Climate, Pollution \& Long-term
Trends}\label{global-change-climate-pollution-long-term-trends}

\subsubsection{Resilience and Recovery of Floral and Nectar Traits under
Acute Heat
Stress}\label{resilience-and-recovery-of-floral-and-nectar-traits-under-acute-heat-stress}

\textbf{Presenter:} Alba Edwards

\textbf{Affiliation:} University of Sussex

\textbf{Authors:} Alba Edwards, Lucy Unwin, Maria Clara Castellanos

Climate change is driving more intense and prolonged heatwaves, imposing
acutely stressful conditions on organisms and ecosystem interactions.
During heatwaves, flowering plants exhibit weakened physiological
function and disrupted reproductive development. Thermal stress can
further diminish the production of floral nectar, which is essential to
pollinator attraction. As a consequence, heat-associated reproductive
losses can have significant consequences for both wild and crop plants,
which are vital for maintaining ecological stability and ensuring food
security. In this study, I investigated the impacts of simulated
heatwaves on floral and nectar traits in the common bean (Phaseolus
vulgaris). Results here indicate that heatwaves can significantly alter
flower and nectar production in the species. Exposure to an extreme
2-day heatwave (daytime 33°C) caused significant reductions in floral
output, flower size, nectar volume and sugar concentration, with the
latter of these traits expressing slow and incomplete recovery. These
findings highlight how even short-lived severe heat events can
negatively modify floral nectar traits, with prolonged effects. Future
studies should adopt field-focused approaches to address the outcomes of
these diminished floral resources on pollinator foraging, to more
intricately determine the consequences of acute heatwaves on plant
reproductive success.

\begin{center}\rule{0.5\linewidth}{0.5pt}\end{center}

\subsubsection{Dedusting herbarium stigmas to uncover historical changes
in plant--pollinator
interactions}\label{dedusting-herbarium-stigmas-to-uncover-historical-changes-in-plantpollinator-interactions}

\textbf{Presenter:} Macarena Marín Rodulfo

\textbf{Affiliation:} University of Granada

\textbf{Authors:} Macarena Marín-Rodulfo¹, Ana Teresa Romero¹, Angela
Cano García¹, Carmen Quesada², Rocío Pérez-Barrales¹

Herbarium collections have become invaluable archives for understanding
ecological and evolutionary processes through time. In this study, we
explore a novel use of herbarium material to analysing pollination
interactions by examining pollen grains on stigmas in specimens of two
Linum species with contrasting pollination systems, the generalised L.
narbonense and the specialised L. suffruticosum s.l., collected across
the Iberian Peninsula since 1899 to 2022. We sampled flowers from sheets
from major Spanish herbaria (MA, BC, VAL, SEV, GDA) and store them in
alcohol 50\% to hydrate stigmas. Then, stigmas were mounted in
fuchsin-stained glycerine jelly for microscopic observation and observed
under x10 magnification to identify pollen grains to determine
intraspecific pollen and heterospecific pollen at the family or
morphotype level. Statistical analyses (GLMMs) using biodiversity
indices reveal patterns of variation in pollen transfer and community
composition through time and space and confirmed the magnitude of
pollination specialization of the species under study. This study
provides unprecedented insights into the historical dynamics of
pollination interactions, as well as methodological basis for future
studies using herbaria to investigate biotic interactions, contributing
to the broader understanding of pollination ecology.

\begin{center}\rule{0.5\linewidth}{0.5pt}\end{center}

\subsubsection{Monitoring Biodiversity in the Genomics Era: Using
herbaria to assess genetic diversity trends across
time}\label{monitoring-biodiversity-in-the-genomics-era-using-herbaria-to-assess-genetic-diversity-trends-across-time}

\textbf{Presenter:} Melissa Viveiros Moniz

\textbf{Affiliation:} University of Granada

\textbf{Authors:} Melissa Viveiros-Moniz, Ana García-Muñoz, Luis Matias,
Mohamed Abdelaziz, Juan Viruel, A. Jesús Muñoz-Pajares

Climate change is having far-reaching consequences on all living beings,
altering ecosystems, habitats, and biodiversity worldwide. Species
distributions are shifting, with alpine plant species being particularly
threatened. Traditional monitoring based on individual counts produce
delayed signals of biodiversity loss and overlook the fact that genetic
diversity is the fundamental basis for evolutionary processes. Here, we
draw attention to the use of genetic diversity in monitoring schemes to
anticipate negative trends in biodiversity by applying two fundamental
methodologies: genomics and the use of herbarium specimens. Genomic
approaches provide a vast amount of data without requiring previous
knowledge of the organism, making them suitable for non-model species.
Meanwhile, herbaria serve as excellent sources of plant material for
comparative studies across time with their chronologically recorded
collection data. Building on these approaches, we investigated temporal
patterns of genetic diversity in endemic alpine plant species from
Sierra Nevada, a region highly vulnerable to climate change. By
combining next-generation sequencing with genomic analyses, we were able
to estimate genetic diversity metrics for each taxon and track changes
over time. Our study highlights the potential of combining genomics and
historical collections to inform conservation strategies in the face of
rapid environmental change.

\begin{center}\rule{0.5\linewidth}{0.5pt}\end{center}

\subsubsection{Effects of climate variability and landscape
modifications on the long-term stability of butterfly communities and
their pollination
interactions}\label{effects-of-climate-variability-and-landscape-modifications-on-the-long-term-stability-of-butterfly-communities-and-their-pollination-interactions}

\textbf{Presenter:} Olivia Gardella

\textbf{Affiliation:} IMEDEA - CSIC

\textbf{Authors:} Olivia Gardella, Pau Colom, Constantí Stefanescu,
Jordi Corbera, Laura Blas \& Amparo Lázaro

Understanding how climate and land-use changes affect pollinators and
their interactions with plants is crucial for predicting ecosystem
responses to anthropogenic pressures. Yet, the mechanisms underlying the
stability of communities and interactions remain poorly understood. We
assessed the effects of climate variability (mean and SD of annual
temperature) and landscape modifications (\% natural areas and landscape
heterogeneity within 2-km buffers) on temporal community stability,
species synchrony, and variance ratio (community variance/sum of
population variances) of butterfly communities and interactions, using
30-year data from seven butterfly communities of the Catalan Butterfly
Monitoring Scheme. Stability metrics, along with landscape and climate
variables were calculated per site within 5-year windows across data
series. We also considered species and interaction diversity as internal
properties linked to stability. The percentage of natural habitats in
the landscape promoted butterfly community stability, while temperature
variability emerged as the main external driver of interaction
instability. Furthermore,

landscape heterogeneity reduced interactions' variance ratio, suggesting
that heterogeneous landscapes enhance interaction stability by reducing
covariances among interactions. Notably, species and interaction
stability strongly increased with diversity. Our findings highlight
biodiversity's pivotal role in sustaining community stability and
ecosystem functions, while cautioning against the destabilizing effects
of climate change and land-use pressures.

\begin{center}\rule{0.5\linewidth}{0.5pt}\end{center}

\subsubsection{Three decades of butterfly--plant interaction turnover
explained by climate and species
loss}\label{three-decades-of-butterflyplant-interaction-turnover-explained-by-climate-and-species-loss}

\textbf{Presenter:} Pau Colom

\textbf{Affiliation:} CREAF

\textbf{Authors:} Pau Colom, Constantí Stefanescu, Jordi Corbera \&
Amparo Lázaro

Understanding the mechanisms behind interaction turnover over long-term
periods is essential to predict how ecological networks respond to
global change. We used a high- resolution dataset of butterfly--plant
interactions spanning 13--29 years in seven Mediterranean communities to
assess how climate fluctuations and community shifts shape interaction
turnover and its components---species turnover and rewiring. Early in
the time series, rewiring explained most interaction turnover, but its
influence declined as species loss reduced the pool of shared partners
between years. Consequently, species turnover became increasingly
dominant, even though communities shifted toward butterfly species with
generalist traits that promote rewiring. Nevertheless, rewiring
intensified in years with stronger temperature fluctuations, when
populations experienced greater shifts in phenology and abundance and
were more likely to rewire. In the context of biodiversity loss, species
turnover increasingly governs interaction dynamics, while the short-term
flexibility provided by rewiring may collapse as communities become
impoverished.

\begin{center}\rule{0.5\linewidth}{0.5pt}\end{center}

\subsubsection{Vulnerability of Oromediterranean pastures to ozone
pollution and atmospheric nitrogen deposition: experimental approaches
for analysing impacts on atmosphere-plant-insect
interactions}\label{vulnerability-of-oromediterranean-pastures-to-ozone-pollution-and-atmospheric-nitrogen-deposition-experimental-approaches-for-analysing-impacts-on-atmosphere-plant-insect-interactions}

\textbf{Presenter:} Sara Campos Saelices

\textbf{Affiliation:} CIEMAT

\textbf{Authors:} Campos-Saelices, S., Prieto-Benítez, S.,
Bermejo-Bermejo, V., González-Fernández, I. \& Cabrero-Sañudo, F.J.

Increased tropospheric ozone (O 3 ) and atmospheric nitrogen (N)
deposition are two environmental problems affecting high-mountain
Mediterranean pastures. When both factors are considered, the critical
thresholds defined for vegetation protection are exceeded in the area,
constituting important stress factors for these communities. While few
experimental O 3 - effects on the vegetative growth of species of these
plant communities have been observed, its impact on flowering-related
variables and reproductive capacity has been demonstrated. N- deposition
can also affect pasture communities by altering their structure and
species composition, as well as by modulating their O 3 -response. This
thesis project presents an experimental design to investigate how an
Oromediterranean pasture community, consisting of seven representative
species, responds to the interaction between O 3 xN, considering four O
3 -levels, ranging from pre-industrial background values to those
predicted throughout this century; and two N-levels, reproducing the
ranges in the area. The experimental assay will be carried out at the
CIEMAT Open Chamber Facility. The effects on variables related to growth
and physiology will be analysed, especially those related to
plant-pollinator relationship. Floral characteristics relating to pre-
and post-pollination will be analysed, as well as pollination and
floral-visitation rates. The effects on life expectancy and insect
growth will be analysed using experimental pollinators.

\begin{center}\rule{0.5\linewidth}{0.5pt}\end{center}

\section{Applied Pollination Ecology}\label{applied-pollination-ecology}

\subsubsection{Monitoring pollinators in the long term: the example of
butterflies and the European Butterfly Monitoring
Schemes}\label{monitoring-pollinators-in-the-long-term-the-example-of-butterflies-and-the-european-butterfly-monitoring-schemes}

\textbf{Presenter:} Constanti Stefanescu

\textbf{Affiliation:} NAtural Sciences Museum of Granollers

\textbf{Authors:} Constantí Stefanescu and Andreu Ubach

In a world fully affected by global change, robust data are needed to
diagnose with certainty trends in biodiversity and, very particularly,
in the different groups of pollinating insects. In this context,
butterflies have emerged as a model group with an enormously popular
methodology for accurately monitoring populations on a large scale, the
so-called Butterfly Monitoring Scheme. Based on a simple census method
that has allowed these projects to be based on citizen science, BMSs
have been implemented in most European countries and have become a very
powerful tool for documenting trends in butterfly populations on the
continent. These databases allow us to explore key aspects such as the
impact of landscape change, local-scale management and climate change on
these insects. With additional effort, it is also possible to obtain
information about the mutualistic networks of butterflies and flowers,
and how they change over time. This talk uses the Catalan BMS to
exemplify some of these aspects. With 32 years of data, 193 butterfly
species and more than 4 million individuals counted, it has become a
reference for studying how Mediterranean ecosystems are changing. We
conclude by exploring new horizons to include other pollinator groups in
monitoring programs.

\begin{center}\rule{0.5\linewidth}{0.5pt}\end{center}

\subsubsection{Do spontaneous ground covers conserve wild pollinators
and enhance crop pollination in apple orchards from northern
Spain?}\label{do-spontaneous-ground-covers-conserve-wild-pollinators-and-enhance-crop-pollination-in-apple-orchards-from-northern-spain}

\textbf{Presenter:} Ángel Plata Sánchez

\textbf{Affiliation:} Departamento Biología de Organismos y Sistemas,
Universidad de Oviedo and Instituto Mixto de Investigación en
Biodiversidad (CSIC-Uo-PA), Oviedo, Asturias, Spain

\textbf{Authors:} Ángel Plata, Teresa Moran-López, Marcos Miñarro,
Daniel García

Promoting non-crop habitats in agroecosystems, such as ground cover
vegetation within crops, may enhance pollinator abundance and diversity
by providing resources that crops lack. These habitats may support broad
pollinator conservation and supply pollinators that spill over to crops
during bloom, enhancing pollination. However, they may also compete with
crops for pollinators. Spill-over and retention can operate
simultaneously, making them difficult to distinguish through approaches
based on species occurrence. Apple orchards offer a valuable model for
evaluating these processes, as apple yield depends on both pollinator
abundance and diversity. In northern Spain, climatic conditions allow
spontaneous ground cover vegetation to persist most of the year with
minimal management, offering a cost-efficient opportunity for growers to
promote ground covers. However, it remains unclear whether such ground
covers effectively conserve wild pollinators and enhance apple
pollination. Here, we characterize flower and pollinator communities in
spontaneous ground covers of twenty-six Asturian apple orchards before,
during, and after apple bloom, and compare them with pollinators
visiting apple flowers. We then assess how flower abundance and
diversity in ground covers shape pollinator communities both within the
covers and on apple trees. Finally, we discuss approaches to infer
whether ground covers drive pollinator spill-over and/or retention.

\begin{center}\rule{0.5\linewidth}{0.5pt}\end{center}

\subsubsection{Assessing nesting patterns of Osmia spp. in almond
orchards across contrasting landscape
contexts}\label{assessing-nesting-patterns-of-osmia-spp.-in-almond-orchards-across-contrasting-landscape-contexts}

\textbf{Presenter:} Gabriel Arbona Taberner

\textbf{Affiliation:} Universitat de les Illes Balears

\textbf{Authors:} Gabriel Arbona, Cayetano Herrera, Andreu Juan, Anna
Traveset, Mar Leza

The decline of wild pollinators threatens crop production, emphasizing
the need to diversify pollination services beyond the managed honeybee.
Solitary Osmia bees are promising alternative pollinators for early
flowering crops due to their high foraging efficiency, ease of
management, and the possibility of synchronizing their emergence with
crop bloom. This study aimed to obtain Osmia cocoons from wild
populations inhabiting almond orchards across Mallorca, representing
contrasting landscape contexts, to examine the structural and ecological
characteristics of their nests. Nesting aids made of natural reed
bundles (300 cavities per site) were installed in 15 orchards. After the
flight season, reeds were dissected and nest traits recorded. Nests and
cocoons with distinct morphologies were detected. In total, 180 nests
and 492 Osmia cocoons were obtained, with mixed-context orchards showing
the highest occupation. Most nests were built in 6-mm reeds, though
typically less than half of the cavity length was used. Nests with fewer
than seven cells exhibited less consistent patterns in cell size and
cocoon weight. Statistical analyses revealed that nest features and the
presence of cleptoparasitic larvae had stronger effects on cocoon
presence than landscape variables. These results provide key insights
for optimizing Osmia-based pollination in almond orchards.

\begin{center}\rule{0.5\linewidth}{0.5pt}\end{center}

\subsubsection{Widespread pollination deficits in pear (Pyrus communis
L.) orchards: the role of pollinators, landscape context and pesticide
risk}\label{widespread-pollination-deficits-in-pear-pyrus-communis-l.-orchards-the-role-of-pollinators-landscape-context-and-pesticide-risk}

\textbf{Presenter:} Lucia Lenzi

\textbf{Affiliation:} University of Bologna

\textbf{Authors:} Lucia Lenzi, Arnan Xavier, Jordi Bosch, Adele Bordoni,
Laura Zavatta, Serena Magagnoli, Agata Morelli, Fabio Sgolastra

European pear (Pyrus communis L.) is an important entomophilous crop,
and most varieties are self- incompatible, therefore strongly dependent
on insect pollinators. However, harsh conditions during bloom and low
sugar content of nectar often lead to low pollinator visitation rates,
causing shortfalls in production. Our aim was to assess pollination
services, detect pollination deficits in pear orchards and analyze the
effects of local factors (pesticide load, orchard management) and
landscape factors (``pollinator-friendly'' cover) on pollinators and
pollination services. Our results confirm the dependence of fruit set on
pollination (mean 37\%). We also report significant pollination deficits
across pear orchards (mean 31\%), and low pollination service (mean
17\%). Most flower visitors were honeybees and Diptera Muscidae, while
wild bees were the least abundant group. However,
``pollinator-friendly'' cover (1.5 km) positively influenced wild bees'
visitation rate. Pollinators had no effect on pollination deficit, but
higher bumblebee visits negatively affected seed set. Pear flowers were
contaminated with at least four pesticides, and pesticide risk had a
negative effect on fruit set. Our results indicate insufficient
pollination services in pear orchards and raise concerns about the
management of pollination provision, highlighting the importance of
semi-natural areas to boost wild bee visits and reduce pesticide
pressure on pollinators.

\begin{center}\rule{0.5\linewidth}{0.5pt}\end{center}

\section{Urban Pollination Ecology}\label{urban-pollination-ecology}

\subsubsection{Thermal buffering ability of butterflies across urban and
natural
environments}\label{thermal-buffering-ability-of-butterflies-across-urban-and-natural-environments}

\textbf{Presenter:} Ashley Tejeda Meneses

\textbf{Affiliation:} University of Barcelona

\textbf{Authors:} Ashley Tejeda Meneses, Pau Colom Montojo, Andrew
Bladon \& Yolanda Melero Cavero

Urbanization alters microclimates, potentially affecting pollinator
activity and plant--pollinator interactions. We examined whether the
thermoregulation of butterflies, key pollinators in Mediterranean
ecosystems, is affected by urbanization. Specifically, we test if urban
populations exhibit enhanced thermal buffering ability compared to
natural ones, and whether this capacity predicts species persistence in
cities. We conducted field surveys in urban parks and surrounding
natural areas in Barcelona, recording air and thoracic temperature of
butterflies, to quantify thermal buffering ability across species and
populations. We did this during their flight period in the bioclimatic
region (March-October) over two years. Preliminary analyses suggest
inter- and intraspecific differences in thermal buffering. Some species
show enhanced thermoregulation in urban areas, while others appear more
vulnerable to urban heat. Variation between populations of the same
species also indicates possible local adaptation or plasticity. Our
results indicate that behavioral thermoregulation is a crucial mechanism
for coping with urban heat islands and a key filter determining which
butterfly species can thrive in them. Such information can help
prioritize conservation actions and guide the management of urban green
spaces. Understanding these patterns helps predict changes in
pollination dynamics under urban warming, as butterfly activity
influences floral visitation and plant reproduction.

\begin{center}\rule{0.5\linewidth}{0.5pt}\end{center}

\subsubsection{Integrating conservation and public engagement through
pollinator gardens: lessons from the Botanical Garden of
Bologna}\label{integrating-conservation-and-public-engagement-through-pollinator-gardens-lessons-from-the-botanical-garden-of-bologna}

\textbf{Presenter:} Marta Barberis

\textbf{Affiliation:} University of Bologna

\textbf{Authors:} Marta Barberis, Fortunato Fulvio Bitonto, Nicola
Herrmann Lothar, Ioannis Mondin, Costanza Viglianisi, Mariacristina
Laureti, Martina Capacci, Silvia Del Vecchio, Umberto Mossetti, Laura
Bortolotti, Annalisa Managlia, Marta Galloni

Pollinators play a crucial role in maintaining biodiversity and ensuring
the productivity of natural and agricultural ecosystems. However, over
the past decades, they have been declining due to habitat loss,
pesticide use, and climate change. The implementation of pollinator
gardens represents an effective action for restoring urban green spaces
while raising public awareness about the topic. An example is
represented by the Pollinator Garden established at the Botanical Garden
of Bologna as part of the LIFE 4 Pollinators project (LIFE18
GIE/IT/000755). It includes nearly 80 nectar-rich species selected to
ensure continuous floral resources throughout the seasons, organized in
flowerbeds representative of the main floral morphologies (sensu Faegri
and Van der Pijl). During the first two years following establishment,
flower- insect interactions were monitored weekly from March to December
by walking a transect running along the perimeter of each flowerbed. The
no. of recorded interactions was 6868, observed during 78 monitoring
days. Alongside, the total number of pollination units per plant was
counted, for a total no. exceeding 124,000. Comparison of network
indices revealed increased connectance, links per species, and
nestedness. Here, we present the results obtained from network analysis,
the concept beyond design, as well as challenges and opportunities
encountered.

\begin{center}\rule{0.5\linewidth}{0.5pt}\end{center}

\section{Citizen Science}\label{citizen-science}

\subsubsection{Easy pollinator learning with
PreguntadoR}\label{easy-pollinator-learning-with-preguntador}

\textbf{Presenter:} Esther Funes-Ligero

\textbf{Affiliation:} Universidad de Granada (UGR)

\textbf{Authors:} Esther Funes-Ligero, Mohamed Abdelaziz, A. Jesús
Muñoz-Pajares.

Learning how to identify pollinators is often hard for students. This
key skill usually requires lot of time from teachers and repeated trips
outside. To fix this problem in learning taxonomy, we created
PreguntadoR, a special app built entirely with R for interactive
learning. PreguntadoR takes the tough job of learning insect body parts
and makes it quick and fun. We designed it to give users a structured,
but very adaptable, place to gain both basic and advanced knowledge. The
app uses game-like features with its different settings and personalized
exercises. When users practice with specific tasks, they see the
important visual signs and classification groups they need for correct
identification over and over. This system makes sure the key differences
in shapes stay in their memory fast. So, PreguntadoR serves as a strong
digital helper for teachers, supporting classes and independent study,
and opens up the complex world of pollinator taxonomy for everyone.

\begin{center}\rule{0.5\linewidth}{0.5pt}\end{center}

\subsubsection{Using Citizen Science to Expand Plant-Animal Interaction
Data in the
Pyrenees}\label{using-citizen-science-to-expand-plant-animal-interaction-data-in-the-pyrenees}

\textbf{Presenter:} Oriane Hidalgo

\textbf{Affiliation:} Institut Botànic de Barcelona

\textbf{Authors:} Iván Pérez Lorenzo, Leonardo Platania, Luis Palazzesi,
Jaume Pellicer, Oriane Hidalgo

Plant-animal mutualisms have strongly shaped the evolutionary
trajectories of both lineages, yet research often centers on conspicuous
pollination systems (e.g., orchids) and narrow spatio- temporal scales
(e.g., single-site daytime monitoring). This bias is acute in the
megadiverse, globally distributed family Asteraceae, perceived as
generalist. In this context, citizen-science platforms could offer a
complementary source of observations, though their usefulness for
interaction ecology needs careful evaluation. Here, we assess the value
of citizen-science records for studying interactions between Asteraceae
and invertebrates (inc. Insecta, Arachnida and Gastropoda) in the
Pyrenees. We built a curated dataset combining iNaturalist observations
with targeted field sampling at monitored sites, currently encompassing
c.~14,000 records of plant-animal interactions. We describe the
taxonomic composition and spatial distribution of the dataset, and
identify common sources of bias. We also illustrate practical
applications, including generating reference lists of invertebrate
species for training automated identification tools, and constructing
plant-animal interaction networks. Our results show that reviewed
citizen-science observations, when complemented with focused fieldwork,
can substantially increase the amount and diversity of interaction data
available. This integrated approach provides an efficient way to improve
biodiversity monitoring and to support plant-focused analyses of
interaction networks at multiple scales.

\begin{center}\rule{0.5\linewidth}{0.5pt}\end{center}

\section{Pollination, herbivory, microbiota and
pathogens}\label{pollination-herbivory-microbiota-and-pathogens}

\subsubsection{Developing Functional Profiles for Wild and Managed
Pollinator Gut
Microbiomes}\label{developing-functional-profiles-for-wild-and-managed-pollinator-gut-microbiomes}

\textbf{Presenter:} Christian Gostout

\textbf{Affiliation:} BC3 Basque Center for Climate Change

\textbf{Authors:} Christian Gostout, Luis J. Chueca, Xabier
Salgado-Irazabal, Estefanía Tobajas, Jennifer Rose, Brais Hermosilla,
Montserrat Muriana, Celia Baigorri, Jon Poza, Ainhoa Magrach

The gut microbiome of wild pollinators plays an important role in
pollinator health and resilience. The composition of pollinator gut
microbiota has been characterized using metabarcoding, and more
recently, metagenomic approaches for a limited number of species,
especially commercially used pollinators. These approaches have allowed
the observation of changes in the composition of the gut microbiome in
response to certain environmental factors, but the implications of these
changes for gene expression and metabolic pathways, and in turn,
pollinator health, are virtually unknown. We move beyond studies focused
on the composition of gut microbiomes to focus on their functions using
metatranscriptomic analyses of RNA extracted from the gut of the wild
pollinator, Bombus pascuorum, and the managed species, Apis mellifera.
We show that taxonomic profiles can be used to make initial inferences
of function, and that metatranscriptomics can confirm functional
expression. Across 122 specimens, we detected an overrepresentation of
gene expression transcripts linked to specific metabolic pathways,
including the breakdown of sugars and complex polysaccharides. Our study
expands microbiome functional studies to wild pollinators, and creates
new commentary on their interactions with managed species. We present an
initial perspective on our study approach and outlook, including a look
at preliminary results.

\begin{center}\rule{0.5\linewidth}{0.5pt}\end{center}

\subsubsection{Pathogen Transmission Through the Bee--Flower Network in
Urban
Ecosystems}\label{pathogen-transmission-through-the-beeflower-network-in-urban-ecosystems}

\textbf{Presenter:} Giovanni Cilia

\textbf{Affiliation:} CREA-AA

\textbf{Authors:} Giovanni Cilia, Dario Scalambra, Rosa Ranalli, Laura
Zavatta, Marta Galloni

Urban environments, with their concentrated floral resources and complex
pollinator communities, create plant--pollinator--pathogen networks that
can facilitate pathogen transmission. This study investigated these
interactions in two urban parks in Northern Italy by sampling wild bee
females, their pollen loads, and the last visited flowers (377 samples
per matrix, April--September 2024). Molecular analyses revealed that
pathogens were deeply surrounded within the network structure: DWV
(45.2\%) and Nosema ceranae (41.3\%) circulated across bees, flowers,
and pollen, confirming the bidirectional movement of pathogens through
shared floral resources. DWV reached 73.3\% prevalence in bees and loads
of 3.6 × 10¹² copies, while N. ceranae was most common on flowers
(41.5\%) and in pollen (38.7\%), indicating that flowers act as
persistent environmental reservoirs. Seasonal patterns showed increased
prevalence in bees during warm periods but stable pathogen abundance in
flowers and pollen, suggesting continuous environmental contamination
even when bee infection levels fluctuate. Pollen identification
reconstructed individual foraging networks, revealing that bees visiting
a richer diversity of plant species had lower probabilities of pathogen
presence and co-infection. Overall, these findings demonstrate that
urban plant--pollinator networks function as tightly interconnected
systems for pathogen exchange, highlighting the need to integrate
pollinator health into urban ecological planning.

\begin{center}\rule{0.5\linewidth}{0.5pt}\end{center}

\subsubsection{Unmasking microbial associations behind insect-attractant
scents in the mucilage droplets of sticky carnivorous
plants}\label{unmasking-microbial-associations-behind-insect-attractant-scents-in-the-mucilage-droplets-of-sticky-carnivorous-plants}

\textbf{Presenter:} Celia Vaca Benito

\textbf{Affiliation:} University of Cádiz

\textbf{Authors:} Celia Vaca-Benito1, María Salces-Castellano1, Ceferino
Carrera2,3, Irene Punta3, Belén Floriano4 \& Fernando Ojeda1

Nectar-dwelling microbes modify the floral scents that attract insect
pollinators 1,2 . Similar to floral nectar, the mucilage droplets on the
leaf-traps of sticky carnivorous plants (e.g., Drosera, Drosophyllum)
emit olfactory signals -- often mimicking floral or fruit scents -- to
lure prey insects 3 . Although bacteria and fungi have been documented
in the mucilage of sundews (Drosera spp.) 4 , the microbiome of
Drosophyllum lusitanicum has not been examined, nor has the potential
contribution of these microbes to leaf-trap scent. We investigated the
mucilage microbiomes of Drosophyllum lusitanicum (nine populations) and
two Drosera species, D. intermedia (six populations) and D. rotundifolia
(three populations), using metabarcoding of the 16S rRNA gene (bacteria)
and the ITS region (fungi). We also characterized their volatilomes
(volatile organic compound profiles) using direct thermal
desorption--gas chromatography/mass spectrometry (TD-GC/MS) 5 . To
assess links between scent profiles and microbial communities, we
applied co-inertia analyses comparing PCoA ordinations of the bacterial
and fungal datasets with those of the volatilome across all 18
populations. We found a significant common structure between the
microbiome and the volatilome for fungi, but not for bacteria,
suggesting that fungi may play a more prominent role in shaping the
luring scent of the mucilage droplets in sticky carnivorous plants.

\begin{center}\rule{0.5\linewidth}{0.5pt}\end{center}

\subsubsection{How do herbivores modulate floral phenotypic plasticity
to abiotic
conditions?}\label{how-do-herbivores-modulate-floral-phenotypic-plasticity-to-abiotic-conditions}

\textbf{Presenter:} Violeta Quiroga Álvarez

\textbf{Affiliation:} Estación Experimental de Zonas Áridas (EEZA-CSIC)
\& Universidad de Granada (UGR)

\textbf{Authors:} Violeta Quiroga-Álvarez, Adela González-Megías,
Cristina Armas, José María Gómez \& Francisco Perfectti

Phenotypic plasticity is the ability of a genotype of producing
alternative phenotypes when exposing to different environments, so it
can be triggered by both biotic and abiotic factors. A typical plastic
response in plants occurs in response to herbivore attack with the
induction of defenses, but the role of the herbivores as modulators of
the plastic response of the plant to abiotic conditions has been seldom
studied. In this study, we experimentally explore the effect of damage
by florivores and folivores on the occurrence and intensity of floral
phenotypic plasticity of Moricandia arvensis (Brassicaceae) under two
contrasting abiotic conditions. In nature, this mustard species blooms
in two contrasting environments, facing mild and wet conditions during
spring, and hot and dry during summer. In response to these
environmental changes, the same individual is plastic for floral traits.
Our preliminary results show that plants attacked by each type of
herbivores retain the capacity to flower during summer conditions,
expressing plasticity for floral traits. These herbivores limit the
plastic response of the plant to the abiotic conditions. This study
highlights the complex interaction between biotic and abiotic stressors
and their combined effect for the evolution of plasticity in M.
arvensis.

\begin{center}\rule{0.5\linewidth}{0.5pt}\end{center}




\end{document}
